\section{Conclusion and Future Works}
In this study we have proposed deep learning approach to build classifiers for tomato plant diseases recognition and the related sensitivity analysis on PlantVillage dataset. Our results confirms what claimed in \cite{ref11} that deep models perform well on this task. Furthermore we provide an improved dataset to overcome the problem of homogeneous backgrounds using the sensitivity analysis as highlighted during the work.
\\\indent
A final remark should be made about the results obtained by the Occlusion method proposed in \cite{ref11}. We have tried to reproduce the same results with our models, but we have not been able at all to come up with the same conclusion about the independence between the class prediction and the background.
\\\indent
In order to develop further assess the performances of our final model on the Random Background Image dataset, we have created a new test dataset containing $256$ images found on Internet. The average accuracy we have obtained is $29.8\%$. Moreover, we have tried to train a VGG11 model on Random Background Image dataset and tested it against the images coming from Internet; the average accuracy we have obtained is $35.8\%$.
We think this result is so poor if compared with the results on Original dataset because no one validated this test dataset. In order to improve this work we suggest to use a dataset of images taken from plantations and validated by a domain expert.