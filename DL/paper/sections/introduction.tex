\section{Introduction}
The advances of science and technology in history have given the possibility to produce enough food to meet the demand of more than 7 billion people. However, food provisioning is threatened by a number of factors such as climate change \cite{ref1}, the decline in pollinators \cite{ref2}, plant diseases \cite{ref3}, and others. Plant diseases are not only a threat to food security at the global scale, but can also have disastrous consequences for smallholder farmers whose livelihoods depend on healthy crops. In the developing world, more than 80 percent of the agricultural production is generated by smallholder farmers \cite{ref5}, and reports of yield loss of more than 50\% due to pests and diseases are common \cite{ref6}.
Various efforts have been developed to prevent crop loss due to diseases based on pesticide usage. Independent of the approach, identifying a disease correctly when it first appears is a crucial step for effective and efficient disease management \cite{ref8}.
\\\indent
Historically, diseases identification has been supported by agricultural experts directly on the field. In recently times, these efforts have been additionally supported by leveraging the increasing of Internet penetration worldwide with on-line diagnoses and the tools based on mobile phones, taking advantage of the rapid uptake of mobile phones technology in all parts of the world \cite{ref9}. These factors, together with advances in computer vision and machine learning, lead to a situation where disease diagnosis based on automated image classification, if technically feasible, can be made available at an unprecedented scale and cost-effectiveness.
\\\indent
On this line, our work focuses first on a Deep Learning approach to disease identification task of 10 tomato plant classes (1 healthy and 9 diseases) using PlantVillage \cite{PlantVillage} dataset (leaves images) and secondly on the targeted sensitivity analysis of the dataset which has been, in fact, used in state-of-the-art related works \cite{ref11, ref10}. The rationale behind the choice of a particular specie of plants (i.e., tomato plants) is that farmers do know what their plantations are about, hence we exploit this fact as prior knowledge.
\\\indent
As first step we reproduce the experiments of \cite{ref11} improving their performance results. Then, we show how the learned models perform against variations of input images. This analysis is further supported by the usage of visualization techniques. Finally, on that basis, we conduct a sensitivity analysis of the dataset by building ad hoc variations of it. The analysis of these variations gives insights on actual robustness of the dataset. Indeed, the experiments are pursued keeping in mind realistic deployment environments in which the prediction phase will be performed (i.e., images taken from plantations, greenhouses, and so on). On this assumption we show that the dataset has some not negligible limitations. In light of this we propose a reasonable image augmentation choice that lets the dataset be more robust to various deployment environments.
\vspace{-5pt}