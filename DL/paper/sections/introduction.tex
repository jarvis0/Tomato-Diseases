\section{Introduction}
The advances of science and technology in history have given the possibility to produce enough food to meet the demand of more than 7 billion people. However, food provisioning is threatened by a number of factors such as climate change (Tai et al., 2014), the decline in pollinators (Report of the Plenary of the Intergovernmental Science-PolicyPlatform on Biodiversity Ecosystem and Services on the work of its fourth session, 2016), plant diseases (Strange and Scott, 2005), and others. Thus, these factors cause direct impacts on the population, such as economic, health, and livelihood impacts \cite{AuthorsMasnzonis}. Plant diseases are not only a threat to food security at the global scale, but can also have disastrous consequences for smallholder farmers whose livelihoods depend on healthy crops. In the developing world, more than 80 percent of the agricultural production is generated by smallholder farmers (UNEP, 2013), and reports of yield loss of more than 50\% due to pests and diseases are common (Harvey et al., 2014). Furthermore, the largest fraction of hungry people (50\%) live in smallholder farming households (Sanchez and Swaminathan, 2005), making smallholder farmers a group that is particularly vulnerable to pathogen-derived disruptions in food supply.
Various efforts have been developed to prevent crop loss due to diseases based on pesticide usage. Independent of the approach, identifying a disease correctly when it first appears is a crucial step for effective and efficient disease management (Ehler, 2006).
\\Historically, disease identification has been supported by agricultural extension organizations or other institutions, such as local plant clinics that have provided expertise support on the field. In recently times, these efforts have been additionally supported by leveraging the increasing of Internet penetration worldwide with on-line diagnoses and the tools based on mobile phones, taking advantage of the rapid uptake of mobile phones technology in all parts of the world (ITU, 2015). These factors, combined together, lead to a situation where disease diagnosis based on automated image classification, if technically feasible, can be made available at an unprecedented scale.
\par On this line, our work focuses first on a deep learning approach to the image classification task of ten tomato plant classes (one healthy and nine diseases) using the PlantVillage dataset\cite{plantvillage} and secondly on the targeted sensitivity analysis of the dataset which has been, in fact, used in state-of-the-art works about the task at hand. As first step we reproduce the experiments of several related works improving their performance results. Then, we show how the learned model responds to input images by using two visualization techniques: GradCam and Occlusion. Finally, on that basis, we make a sensitivity analysis of the original dataset by building ad hoc variations of it. These variations, together with model visualization, give us insights on the dataset actual robustness. Indeed, the experiments are pursued keeping in mind realistic environments in which the prediction phase will be performed (i.e., images taken from plantations, greenhouses, and so on).