\section{Related work}
Several works have been proposed in the literature to plant diseases identification. The classical approach given by the expertise support directly on the field has offered diverse solutions that show outstanding performance, nonetheless they do not provide yet a scalable and cost-effective solution \cite{ref17, ref18, ref16}.
\\\indent
After analysis of their work and investigation presented by the authors of \cite{ref15, ref14}, it has been decided to employ the image processing approach among other laboratory-based approaches. Several handcrafted feature-based methods have been widely applied specifically for image processing. These methods are usually combined with classifiers from machine learning (e.g., Support Vector Machines (SVMs) \cite{ref24}, K-Nearest Neighbors \cite{ref25}, Random Forests \cite{ref26}, and so on). In \cite{ref27}, the authors have presented a survey of well-known conventional methods for handcrafted feature extraction.
\\\indent
The main drawback of these methods regards feature engineering, that is a complex and time-consuming process which needs to be revisited every time according to the problem at hand. Thus, the performance of classifiers depends heavily on the underlying features. For these reason, deep learning has allowed researchers to consider and design systems as a unified and automated process with no \emph{handcrafted} features \cite{ref29}. In particular, Convolutional Neural Networks (CNNs), first introduced in \cite{ref30}, have showed, in fact, how to bind together feature extraction to classification in image recognition by means of LeNet architecture. Tremendous achievements have been made by CNNs in the past few years in image classification and benchmarked, for instance, against ImageNet dataset \cite{ref28}.
\\\indent
The principles of CNNs have spread also to plant diseases identification. The authors of \cite{ref11}, have proposed a comparison between shallow models with combined handcrafted features and deep models for the identification of 9 tomato plant diseases. They have analyzed different models using PlantVillage dataset containing $14,828$ images of cropped leaves put on a table. The best shallow model has achieved $95.47\%$ accuracy using Random Forests, while the best deep model has achieved $99.19\%$ accuracy using GoogLeNet architecture (pre-trained on ImageNet). Furthermore, they have used Occlusion \cite{ref13} as model visualization method and on its basis they claim that backgrounds in images do not influence the prediction of their best model. In other related works \cite{ref33, ref10} and in our, we argue that this is not always true.
\\\indent
The work proposed in \cite{ref10} have focused on developing deep models for the identification of 38 classes (i.e., crop and disease information) using PlantVillage dataset (54,306 images of leaves). They have compared several models having various hyper-parameters and dataset variations (RGB, gray-scale, segmented). Their best result, 99.34\% accuracy, has been achieved by GoogLeNet (pre-trained on ImageNet) employing RGB images. They noticed that their best model, when tested on a set of images (derived from trusted Internet sources) taken under conditions different from the images employed for training, determines a substantial accuracy reduction, to just above 31\%. According to the authors, this limitation is caused by homogeneous background in the images.
\\\indent
A more robust approach has been pursued in \cite{ref33} where the authors proposed a robust deep-learning-based detector for real-time tomato diseases and pests recognition. They have built their own dataset of 5,000 images taken under different conditions and scenarios divided in 9 classes (and the backgrounds class). Then, with the support of experts, they manually annotated the areas of every image containing the disease or pest with a bounding box and class. Finally, they have analyzed several models for object detection and achieving outstanding performances. Interestingly, they criticize PlantVillage dataset since the images it contains have been previously cropped in the field and captured by a camera in the laboratory causing image recognition and object detection on realistic environments images unfeasible. Instead, their work has aimed at dealing with background variations mainly caused by the surrounding areas of plants or the place itself (i.e., images taken from plantations, greenhouses, and so on).
\\\indent
According to the complains that have been made about PlantVillage dataset, our work provides a reasonable heuristic solution to overcome its limitations for what backgrounds are concerned.